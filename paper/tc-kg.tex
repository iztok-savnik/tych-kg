\documentclass[runningheads]{llncs}

\pdfoutput=1
\usepackage{times}
\normalfont
\usepackage{latexsym}
\usepackage[T1]{fontenc}
\usepackage[noend]{algpseudocode}
\usepackage{algorithm}
\usepackage{graphicx}
\usepackage{wrapfig}
\usepackage{verbatim}
\usepackage{amsmath}
\usepackage{amssymb}
\usepackage{stmaryrd}
\usepackage{graphics}
\usepackage{color}
\usepackage{url}

% symbols
\newcommand{\T}{{\cal T}} 
\newcommand{\D}{{\cal D}} 
\newcommand{\I}{{\cal I}} 
\newcommand{\Ii}{{\cal I}_i} 
\newcommand{\Ic}{{\cal I}_c} 
\newcommand{\Ip}{{\cal I}_p} 
\newcommand{\cl}{\text{:}} 
\newcommand{\rdftype}{\text{rdf:type}} 
\newcommand{\rdfsubcl}{\text{rdfs:subClassOf}} 
\newcommand{\rdfsubclpl}{\text{rdfs:subClassOf+}} 
\newcommand{\rdfsubprop}{\text{rdfs:subPropertyOf}} 
\newcommand{\rdfsubproppl}{\text{rdfs:subPropertyOf+}} 
\newcommand{\rdfsdomain}{\text{rdfs:domain}} 
\newcommand{\rdfsrange}{\text{rdfs:range}} 
\newcommand{\finbox}{\phantom{.}\hfill\Box}
\newcommand{\nl}{\hfill\break}
\newcommand{\vb}[1]{\begin{small}\texttt{#1}\end{small}}
\newcommand{\memo}[1]{}
\newcommand{\notes}[1]{\noindent\begin{small}-- \emph{#1}\\\end{small}}

\def\ojoin{\setbox0=\hbox{$\bowtie$}%
  \rule[-.02ex]{.25em}{.4pt}\llap{\rule[\ht0]{.25em}{.4pt}}}
\def\leftouterjoin{\mathbin{\ojoin\mkern-5.8mu\bowtie}}
\def\rightouterjoin{\mathbin{\bowtie\mkern-5.8mu\ojoin}}
\def\fullouterjoin{\mathbin{\ojoin\mkern-5.8mu\bowtie\mkern-5.8mu\ojoin}}

\pagestyle{headings}
\setcounter{tocdepth}{5} %show more in the toc

\begin{document}

\title{Type-checking knowledge graphs}

\author{Iztok Savnik\inst{1} \and Kiyoshi Nitta\inst{2}} 

\authorrunning{I. Savnik et al.}

\institute{Faculty of mathematics, natural sciences and information technologies, \\
  University of Primorska, Slovenia \\
  \email{iztok.savnik@upr.si} \and
  Yahoo Japan Corporation, Tokyo, Japan \\
  \email{knitta@yahoo-corp.jp}}

\maketitle

\begin{abstract}
  We first present a formal view of a knowledge graph. On this basis,
  the type-checking rules are developed to define correct typing
  relationships among the triples of a knowledge graph. We discuss the
  algorithms for verifying the typing relationships against the given
  knowledge graph. Finally, we present the experimental results of
  type-checking the Yago4 knowledge graph.

\keywords{RDF stores \and graph databases \and knowledge graphs \and database statistics \and statistics of graph databases.}
\end{abstract}

%\newpage
%\tableofcontents
%\newpage

\thispagestyle{headings}
\tableofcontents
\newpage

\section{Introduction}

This is intro... \cite{Hogan22KGsSurveyCsur}.

\section{Formal framework}

This section describes the formal view of knowledge graphs 

\section{Typing knowledge graphs}

\subsection{Typing literals}



\subsection{Typing identifiers}

\notes{The following is view from the formalization.}

The set $I$ includes individual identifiers $I_i$, class identifiers
$I_c$ and predicate identifiers $I_p$. Let $i_1,i_2\in\/I$. The
relationship preceeds $\preceq$ on the set $I$ is defined as
follows. If the identifier $i_1$ is more specific than or equal to
$i_2$ with respect to the conceptual schema of a knowledge graph, then
$i_1\preceq\/i_2$.

The relationship $\preceq$ defines a partial ordering of the
identifiers from $I$ that we denote $(I,\preceq)$. As described in the
section on formalization, the class identifiers $I_c$ stand for the
types of individual identifiers $I_i$. Hence, the partial ordering
$(I,\preceq)$ is defined by means of the relationships
rdf:type, rdfs:SubClassOf and rdfs:subPropertyOf. In this way, we
obtain also the isomorphical poset defined on the interpretations of
individual types (classes) using the subsumption relationship
$\subseteq$.

\notes{We now state the above in the realm of the sub-typing relationship.}

\subsubsection{Stored sub-typing of identifiers.}\nl

\notes{Partial ordering defined with triples in a database.}
\notes{The relationships that poset covers are rdf:type, rdfs:subClassOf and rdfs:subPropertyOf. }
\notes{All identifiers included in the relalationship $\preceq_1$.}
\notes{This allows us to separate and also address separately the ssg and subtyping relationship.}
\notes{The relation $\preceq_1$ includes solely the \emph{stored} relationships among identifiers.}
\notes{The relation $\preceq$ is the relation $\preceq_1$ extended with the reflexivity and transitivity.}

\notes{Oportunity to introduce ``mixed'' objects including ground and schema components.}

\noindent Reflecting the one-step relationship rdf:type in $(\I,\preceq)$. 

\begin{equation}
\dfrac{I_1\in\Ii \quad I_2\in\Ic \quad (I_1,\text{rdf:type},I_2)\in \D}
      {I_1 \preceq_1 I_2}
\end{equation}

\noindent Including the one-step relationship rdfs:subClassOf in
$(\I,\preceq)$.

\begin{equation}
\dfrac{I_1,I_2\in\Ic \quad (I_1,\text{rdfs:subClassOf},I_2)\in \D}
      {I_1 \preceq_1 I_2}
\end{equation}

\noindent Including the one-step relationship rdfs:subPropertyOf in
$(\I,\preceq)$.

\begin{equation}
\dfrac{I_1,I_2\in\Ip \quad (I_1,\text{rdfs:subPropertyOf},I_2)\in \D}
      {I_1\preceq_1\/I_2}
\end{equation}

\notes{Show that all identifiers are included.}


\subsubsection{Subtyping identifiers.}\nl

%\medskip\noindent Including the many-step step relationship rdfs:subClassOf in $(I,\preceq)$.
%
%\begin{equation}%\medskip\noindent Including the many-step step relationship rdfs:subClassOf in $(I,\preceq)$.

%\dfrac{i_1,i_2\in\/I_c \land (i_1,\text{rdfs:subClassOf}+,i_2)}
%      {i_1\preceq i_2}
%\end{equation}
%
%\medskip\noindent Including the many-step step relationship rdfs:subPropertyOf in $(I,\preceq)$.
%
%\begin{equation}
%\dfrac{i_1,i_2\in I_p \land (i_1,\text{rdfs:subPropertyOf}+,i_2)}
%      {i_1\preceq i_2}
%\end{equation}

\notes{Relate everything with subsumption poset.}

Generalizing one-step relationship $\preceq_1$ to the relationship
$\preceq$ in $(I,\preceq)$.  $\preceq_1$ is a basis og $\preceq$.

\begin{equation}
\dfrac{I_1,I_2\in\I \quad I_1\preceq_1 I_2}
      {I_1 \preceq I_2}
\end{equation}

\noindent Subtyping is reflexive.

\begin{equation}
\dfrac{S\in \I}
      {S\preceq\/S}
\end{equation}

\noindent The subtype relationship is transitive. We require that the
symbols $S$, $U$ and $T$ are identifiers. Note that $S$ can be an
individual identifier while $U$ and $T$ have to represent
classes.

\begin{equation}
\dfrac{S,U,T\in\I \quad S\preceq\/U \quad U\preceq\/T}
      {S\preceq\/T}   
\end{equation}

\noindent Types include a special type $\top$ that represents the most
general type in the ontology. Every type is more specific than the top
type $\top$.

\begin{equation}
S\preceq\top
\end{equation}


\subsubsection{Typing of identifiers.}

\noindent A base type of an individual identifier $I$ is a type
related to $I$ by the relationship $\preceq_1$. Derivation of base
types of $I$ is defined using the following rule.

\begin{equation}
\dfrac{I\in\Ii \quad C\in\Ic \quad I\preceq_1\/C}
      {I:_1C}
\end{equation}

\noindent There are two possible ways of defining a type of an
identifier. One way is to use the relationship $\preceq$. The other
way is to use existent typing.

All possible types of $I$ include the base types of $I$ and all
types that are more general than the base types. Note that the
relationship $\preceq$ subsumes the relationship $\preceq_1$.

%%% bounded universal type?
%\begin{equation}
%\dfrac{i_1\in I_i \land \forall\/i_2\in I_c(i_1 \preceq i_2)§}
%      {i_1:i_2/\text{deriv}}   
%\end{equation}

\begin{equation}
\dfrac{I\in\Ii \quad C\in\Ic\quad I\preceq\/C}
      {I:C}   
\end{equation}

\noindent The bridge between the typing relation and subtype relation
is provided by adding a new typing rule \cite{Pierce02TypesProgLang}. The
following rule is called \emph{rule of subsumption}.

\begin{equation}
\dfrac{I\in\Ii \quad I:S \quad S\preceq\/T}
      {I:T}    
\end{equation}

% - lub 

%\subsection{lub}
  


\subsection{Intersection type}

The instances of the intersection type $T_1\land\/T_2$ are objects
belonging to both $T_1$ and $T_2$. The type $T_1\land\/T_2$ is the
greatest lower bound of the types $T_1$ and $T_2$. In general,
$\land[T_1\ldots\/T_n]$ is the greatest lower bound of types
$T_1\ldots\/T_n$ \cite{Pierce91IntersectUnion,Pierce96IntersectionTypes}.

\begin{equation}
T_1\land\/T_2\preceq\/T_1    
\end{equation}

\begin{equation}
T_1\land\/T_2\preceq\/T_2  
\end{equation}

\begin{equation}
\land[T_1\ldots\/T_n] \preceq\/T_i  
\end{equation}

If the type $S$ is more specific than the types $T_1\ldots\/T_n$ then
$S$ is more specific then $\land[T_1\ldots\/T_n]$. First, we present
the rule for a pair of types $T_1$ and $T_2$.

\begin{equation}
\dfrac{S\preceq\/T_1\quad\/S\preceq\/\/T_2}
      {S\preceq\/T_1\land\/T_2}  
\end{equation}

\begin{equation}
\dfrac{\text{forall i,\ } S\preceq\/T_i}
      {S\preceq\land[T_1\ldots\/T_n]}  
\end{equation}




\subsection{Union type}

The intersection and union types are dual. This can be seen also from
the rules that are used for each particular type.

The instances from the union type $T_1\lor\/T_2$ are either the
instances of $T_1$ or $T_2$, or the instances of both types. The type
$T_1\lor\/T_2$ is the smallest upper bound of the types $T_1$ and
$T_2$. In general, $\lor[T_1\ldots\/T_n]$ is the smallest upper bound
of types $T_1\ldots\/T_n$ \cite{Pierce90CalculusIntersectUnion}.

\begin{equation}
T_1\preceq\/T_1\lor\/T_2    
\end{equation}

\begin{equation}
T_2\preceq\/T_1\lor\/T_2  
\end{equation}

\begin{equation}
T_i\preceq\/\lor[T_1\ldots\/T_n]  
\end{equation}

If the type $T$ is more general than the types $S_1\ldots\/S_n$ then
$T$ is more general then $\lor[S_1\ldots\/S_n]$. First, we present
the rule for types $T_1$ and $T_2$.

\begin{equation}
\dfrac{S_1\preceq\/T\quad\/S_2\preceq\/\/T}
      {S_1\lor\S_2\preceq\/T}  
\end{equation}

\begin{equation}
\dfrac{\text{forall i,\ } S_i\preceq\/T}
      {\lor[S_1\ldots\/S_n]\preceq\/T}  
\end{equation}



\subsection{Type-checking triples}

% base
\subsubsection{Triples and schema triples.}\nl

\notes{Is the following defined in formalization of KG?}
\notes{Maybe typing of ground, schema triples is presented? Which aspect?}
\notes{Show the complete poset of triples.}
\notes{Define the set of ground triples.}
\notes{Define the set of type triples (schema triples) and the schema graph .}
\notes{Define the stored schema graph.}

\subsubsection{Deriving a base type of a triple.}

\noindent The base type of an individual identifier $i$ is a class $c$
related to $i$ by one-step relationship $\preceq_1$. In terms of the
concepts of a knowledge graph, $c$ and $i$ are related by the
relationship rdf:type.

A base type of a triple $t=(s,p,o)$ is a triple $T=(T_s,T_p,T_o)$ that
includes the base types of $t$'s components. A base type of a triple
is defined by the following rule.

\begin{equation}
\dfrac{s:_1T_o\quad p:_1T_p\quad T_p\preceq\/\text{rdf:Property}\quad o:_1T_o}
      {(s,p,o):_1T_s*T_p*T_o}
\end{equation}

The types of $s$ and $o$ can be any classes $T_s$ and $T_o$ from
${\cal I}_c$, while the type of $p$ has to be a class $T_p$ that is a
subclass of rdf:Property. The typing of a triple $t$ is correct since
the interpretation of $T$ includes $t$. Moreover, the types $T$ that
are derived by the above rule are minimal in the sense that given the
information provided, i.e., the types of $t$'s components, their
interpretations are minimal possible comparing them to the
interpretations of all other derived types of $t$.

%\medskip\noindent Remarks:
%\begin{itemize}
%\item If $i_1,i_2\in\/I_c \land i_1\preceq\/i_2$ then $i_1\sqcap\/j_1\sqcap\/...\sqcap\/j_k\preceq i_2$.
%\end{itemize}


\subsubsection{Deriving a top type of a triple.}

\noindent The following rule is a judgment for a top type of a
concrete triple $t=(s,p,o)$. A top type of a triple $t$ is the most
specific type from the stored schema graph which interpretation
includes $t$.

We first find the schema triples for a given predicate $p$. The set of
stored schema triples is constructed by selecting the most specific
schema triples with a predicate that is more general then $p$.

\begin{equation}
\begin{array}{l}
S_0 = \{(T_s,p',T_o)\ |\ p'\succeq\/p \land (p',\text{dom},T_s)\in\/g \land (p',\text{rng},T_o)\in\/g\ \land \\
\phantom{S = \{(T_s,p',T_o)\ |} \not\exists\/p''(p''\preceq\/p'\land(p'',\text{dom},T_s)\in\/g\land(p'',\text{rng},T_o)\in\/g)\}
\end{array}
\end{equation}

Generator view of rules: Just describe the properties of
pre-conditions and conclusions.

\begin{equation}
\dfrac{T\in\/\text{ssg}\quad p\preceq\/T_p\quad  
       \text{for all\ } T'\in\/\text{ssg},\ T'\succ\/T \lor p\succ\/T'_p\lor T'_p\succ\/T_p}
      {(s,p,o):_2\/T}
\end{equation}

\begin{equation}
\dfrac{T\in\/\text{ssg}\quad\/t:_1T_1\quad\/T_1\preceq\/T\quad  
       \not\exists\/S\in\/\text{ssg},\ S\prec\/T \land T_1\preceq\/S}
      {(s,p,o):_2\/T}
\end{equation}

The first two premises require that the type $T$ is an element of the
stored schema graph, and the predicate of $T$, i.e., $T_p$, is more
general than the predicate $p$ of the input triple $(s,p,o)$.

The last premise in the above rule requires that the top type $T$ is
the least general type including a predicate equal or more general to
$p$. The condition can be better understood in the existential form:
$\not\exists\/T'\in \text{ssg}: T'\preceq\/T\land
p\preceq\/T'_p\preceq\/T_p$.

Note that the rule is not linked to the $t$'s components $s$ and
$o$ in any way. This means that $s\preceq\/T_S$ and $o\preceq\/T_O$
may not hold.\footnote{Does it make sense to add the conditions? Further, at
this point the type errors can be catched.}

From the other point of view, the schema triples are obtained from the
inherited values of the predicates rdfs:domain and rdfs:range. The
inherited values have to be the closest when traveling from property
$p$ towards the more general properties. Note that multiple different
schema triples are possible only in the case of multiple inheritance.


\subsubsection{Typing a triple.}\nl

\notes{Why the type selected from ssg? }
\notes{How (conceptually) types from ssg are selected?}

The type of a triple $t=(s,p,o)$ is computed by first deriving the
base type $T$ and the top type $S$ of $t$. Then, we check if $S$ is
reachable from $T$ through the sub-class and sub-property hierarchies,
i.e., $T\preceq\/S$.

\begin{equation}
\dfrac{(s,p,o):_1T\quad (s,p,o):_2S\quad (T\preceq\/S)}
      {(s,p,o):S}
\end{equation}

\notes{How to compute $T\preceq\/S$? Refer to position where we have a description.}
\notes{How to gather a complete type of $t$ including different
$S\in\/\text{sgg}$? Union of selected $S$'s... this is a
\emph{complete} type. It does make sense.}

%The alternative types are joined by using the union type contructor to represent the type of $(s,p,o)$.
%
%\begin{equation}
%S = \bigsqcup_{T\in\/S_0}T
%\end{equation}
%
%The final type $S$ of a triple $(s,p,o)$ is composed of the previously selected alternative stored schema triples that are reachable from the base types of the triple. The presented procedure for the computation of $S$ is further referrred to as ssg-type(p) where $p$ is a parameter property.

\noindent
\notes{Order the possible derivations, gatherings (groupings) ... of types.}
\notes{How to derive all possible types of a triple? How to integrate them using union and intersection types?}
\notes{How to derive types of a triple deriving in some specific direction? For example, the cover (lub) type of a triple? The most specific type (base type)?}

\noindent
\notes{Possible diagnoses.}
\notes{Components not related to a top type of a triple?}
\notes{Components related to sub-types of a top type?}
\notes{Above pertain to all components.}


\subsection{Typing a graph.}\nl

\notes{What is a type of a graph?}
\notes{A type of a graph is a graph!}
\notes{It includes a set of schema triples forming a schema graph.}

\notes{Typing a graph bottom-up?}
\notes{Checking that all the triples are of correct types.}


\subsubsection{Typing a schema triple.}\nl

\notes{What can be checked?}
\notes{Is a schema triple properly related to the super-classes and types of components.}
\notes{Consistency of the placement of a class in an ontology. What is this?}
\notes{A class or predicate component not related to other classes?}
\notes{A class or predicate component attached to ``conflicting'' set of classes? What can be detected?}
\notes{@kiyoshi Do you see any other examples?}
\notes{}



\section{Empirical analysis}

\section{Conclusions}



% \section{Acknowledgments}
%
%The authors acknowledge the financial support from the Slovenian Research Agency (research core funding No. P1-00383).

\bibliographystyle{abbrv}
\bibliography{biblio}

\end{document}



= rules for identifiers
- assumtions
. suppose we have I, a set of identifiers from g
. suppose we have a 

- type-check literals


.................before 05/27/22....................

= (task) general picture
- we have graph pattern 
. a set of triples including consts and vars
- we derive types of TPs



- we derive types of GPs


= (task) determine a type of a TP
. type of TP without a predicate? use rdf:predicate.
. type of TP without the schema? use generalizations.
. type of TP with predicate? use a triple type defined as the schema.
. type of TP can be \cup of triple types (defined in the schema).


= (task) compute a type of a GP
. meeting points in GP: use \cap type constructor
. \cap of \cup of triple types can select one of triple types
. one triple type is selected by other oper of \cap


= (task) compute the type of a concrete triple
- we need a type of a concrete triple, e.g. for distribution
. if we have schema triple than it can be utilized
. obtain the type of each constant
. schema type, or, the most specific type?
