\documentclass[runningheads]{llncs}

\pdfoutput=1
\usepackage{times}
\normalfont
\usepackage{latexsym}
\usepackage[T1]{fontenc}
\usepackage[noend]{algpseudocode}
\usepackage{algorithm}
\usepackage{graphicx}
\usepackage{wrapfig}
\usepackage{verbatim}
\usepackage{amsmath}
\usepackage{amssymb}
\usepackage{stmaryrd}
\usepackage{graphics}
\usepackage{color}
\usepackage{url}

% symbols
\newcommand{\s}{\scriptscriptstyle\ \,} 
\newcommand{\ul}{\underline} 
\newcommand{\ol}{\overline} 
\newcommand{\darr}{\downarrow} 
\newcommand{\uarr}{\uparrow} 
\newcommand{\Darr}{\Downarrow} 
\newcommand{\Uarr}{\Uparrow} 
\newcommand{\llb}{\llbracket}
\newcommand{\rrb}{\rrbracket}
\newcommand{\vdasharr}{%
    \mathrel{%
        \vdash\hspace*{-5pt}%
        \raisebox{2.3pt}{\scalebox{.5}{\(\blacktriangleright\)}}%
    }%
}\newcommand{\D}{{\Delta}} 
\newcommand{\U}{{\cal U}} 
\newcommand{\B}{{\cal B}} 
\newcommand{\Lt}{{\cal L}} 
\newcommand{\St}{{\cal St}} 
\newcommand{\Pt}{{\cal Pt}} 
\newcommand{\Ot}{{\cal Ot}} 
\newcommand{\I}{{\cal I}} 
\newcommand{\Ii}{{\cal I}_i} 
\newcommand{\Ic}{{\cal I}_i} 
\newcommand{\Ip}{{\cal I}_p}
\newcommand{\V}{{\cal V}} 
\newcommand{\Vl}{{\cal V}_l} 
\newcommand{\Vi}{{\cal V}_i} 
\newcommand{\Vt}{{\cal V}_t} 
\newcommand{\Vp}{{\cal V}_p}
\newcommand{\T}{{\cal T}}
\newcommand{\Ti}{{\cal T}_i}
\newcommand{\Tt}{{\cal T}_t}
\newcommand{\cl}{\text{:}} 
\newcommand{\rdftype}{\text{rdf:type}} 
\newcommand{\rdfsubcl}{\text{rdfs:subClassOf}} 
\newcommand{\rdfsubclpl}{\text{rdfs:subClassOf+}} 
\newcommand{\rdfsubprop}{\text{rdfs:subPropertyOf}} 
\newcommand{\rdfsubproppl}{\text{rdfs:subPropertyOf+}} 
\newcommand{\rdfsdomain}{\text{rdfs:domain}} 
\newcommand{\rdfsrange}{\text{rdfs:range}} 
\newcommand{\finbox}{\phantom{.}\hfill\Box}
\newcommand{\nl}{\hfill\break}
\newcommand{\vb}[1]{\begin{small}\texttt{#1}\end{small}}
\newcommand{\lub}{l}
\newcommand{\memo}[1]{}
\newcommand{\notes}[1]{\noindent\begin{small}-- \emph{#1}\hfill\break\end{small}}
\newcommand{\nnotes}[1]{\indent\begin{small}-- \emph{#1}\hfill\break\end{small}}
\newcommand{\nnnotes}[1]{\indent\indent\begin{small}-- \emph{#1}\hfill\break\end{small}}
\newcommand{\anotes}[1]{\indent\begin{small}\phantom{-- }\emph{#1}\hfill\break\end{small}}
\newcommand{\ennotes}[1]{\indent\begin{small}-- \emph{#1}\hfill\end{small}}
 
\def\ojoin{\setbox0=\hbox{$\bowtie$}%
  \rule[-.02ex]{.25em}{.4pt}\llap{\rule[\ht0]{.25em}{.4pt}}}
\def\leftouterjoin{\mathbin{\ojoin\mkern-5.8mu\bowtie}}
\def\rightouterjoin{\mathbin{\bowtie\mkern-5.8mu\ojoin}}
\def\fullouterjoin{\mathbin{\ojoin\mkern-5.8mu\bowtie\mkern-5.8mu\ojoin}}

\pagestyle{headings}
\setcounter{tocdepth}{5} %show more in the toc

\begin{document}

\title{DRAFT! \\ On typing knowledge graphs}

\author{Iztok Savnik\inst{1}} 

\authorrunning{I. Savnik}

\institute{Department of computer science, \\
  Faculty of mathematics, natural sciences and information technologies, \\
  University of Primorska, Slovenia \\
  \email{iztok.savnik@upr.si}}

\maketitle

\begin{abstract}
  The problem addressed in this paper is typing a single ground
  triple~$t$ from a knowledge graph. First, the \emph{ground type} of
  $t$ is inferred bottom-up, from the stored typing of individual
  entities which are the components of $t$. The ground type is further
  minimized and then generalized to obtain a minimal upper bound (MUB)
  type. The MUB type is an appropriate starting point for exploring
  its relations to the conceptual schema of a knowledge graph. Second,
  the \emph{schema type} of $t$ is inferred from the conceptual schema
  based on the predicate of $t$. All valid schema types of $t$ are
  gathered and then minimized to obtain a schema type with a minimal
  interpretation. Finally, the minimal schema type is filtered by
  relating it to the MUB ground type via the subtype relation to
  compute the final schema type of $t$.

  \keywords{knowledge representation \and knowledge graphs \and type systems .}
\end{abstract}

% \newpage
%\tableofcontents
%\newpage

\thispagestyle{headings}
%\tableofcontents
%\newpage




\section{Introduction}

A knowledge graph (abbr.. KG) includes a data and knowledge
representation language in the form of a graph. It is the
dictionaries, such as RDF-Schema \cite{rdfschema}, which attach
meanings to the edges of a graph, turning a graph into a language for
the representation of data and knowledge. In comparison to the data
and program structures used in programming languages
\cite{Pierce2002,Hindley1997}, the conceptual schema of a KG is more
expressive: in principle, any data and program structure can be
represented as a graph. The expressive power comes from logic that
stands behind a KG, where a graph is a set of named relations, and
triples are logical statements. Let us depict some of the features of
the data and knowledge representation language of a KG.

First, a KG includes predefined classes and predicates that are
organized into taxonomies \cite{Baader2002}. Second, typing of ground
identifiers, as well as typing of the predicates with the domain and
range types, is stored in a KG. Further, similarly to the \emph{roles}
\cite{Brachman1985} of a knowledge base, the predicates of a KG are
treated as classes (types) that are included in a taxonomy of
predicates. However, they also act as individual entities described
with additional predicates. Finally, the predicates are inherited
through the taxonomy of classes, and, from the other point of view,
the domain and range types of the predicates are inherited through the
hierarchy of predicates.

Typing a knowledge graph requires a framework for the definition of
rules, which is different from the framework for classical typing of
programming languages. While typing in programming languages is based
on the syntactical composition of a program, a knowledge graph is
built from simple triples. Triples can form rich semantic structures
depending on the kinds of entities and values. The kind of object,
for example, can be either a ground value or a class. Typing of a KG
must, therefore, be able to grasp the kinds of objects in the rules,
and, in many cases, need to be guided by a procedure.

From a logic perspective, the ground entities and triples represent
an ABox, while the schema of KG is a TBox and represents logical
statements about classes and binary relations. The syntactical
structures that we use for types are the following. First, we have 
the classes and predicates, the types of identifiers, that are
identifiers themselves. Second, the type of triples is a product type,
a triple itself. Next, the $\land$ and $\lor$ types capture many
requirements of the KG domain very naturally. For instance, entities
belong to multiple classes simultaneously, or a domain of a predicate
is two different classes. Finally, the taxonomies of classes and
predicates are hard to grasp structurally; rather, posets can be
manipulated with algebraic operations.

The paper presents a study of typing a single ground triple $t$. We
propose to infer first the ground type of $t$ using the stored typing
of identifiers. The ground type is minimized and then generalized to
the minimal upper bound (MUB) type. The MUB type is an appropriate
starting point to search for the triple types from a KG conceptual
schema.

The final triple type of $t=(s,p,o)$ is selected from the triple types
provided by a conceptual schema stored in a KG. Using the predicate
$p$ as the starting point, we choose first all valid triple types
including a predicate $p$ or one of $p$'s super-predicates. The set of
selected triple types is minimized to obtain a set of minimal and
unrelated triple types, the candidates for the type of $t$.

The appropriate candidate triple types are selected by relating the
candidates to the MUB ground type by means of a subtype relation. The
disjunctively bound candidate triple types that are related to the MUB
ground type via subtype relation form the final type of~$t$.

The contributions of the presented research are as follows. First, to
our knowledge, this is the first proposal for typing a knowledge graph
that encompasses a complete KG. The existing approaches to typing a KG
deal with the particular problems of typing a KG. Second, the
presented framework for typing ground triples can serve conveniently
for the implementation of type-checking the stored typing of a KG. The
analysis of typing an individual triple spans from the ground types at
the lower levels of ontology to the minimal schema types at the upper
levels of ontology. Finally, we show that the triple types can be used
to disambiguate the sense of a predicate with multiple meanings.

...\\





%----------------------------------------------------------------
\memo{
%% Intro to KGs... what kind of language is it?
\notes{Introduction to knowledge graphs (KG)... \cite{Hogan2022,Ehrlinger2016}.}
\notes{KGs are becoming knowledge bases (KB)...}
\notes{What are the structural characteristics of KBs?}
\notes{What KBs can represent that (classical) data models can not.}
\notes{Relations between the knowledge bases and KGs.}

%% KG data model - knowledge representation language
\notes{What is structural part of KG?}
\notes{KG domain is complex because of the rich modelling constructs of a KR language.}
\notes{The static structure of a KG is built from the identifiers that are bound into triples.}
\notes{The identifiers and triples represent the basic structures of a KG, i.e., the values stored in a KG.}
\notes{They are the values of languages defined on top of KGs, such as the basic graph-patterns, SparQL queries, and if-then rules.}
\notes{Insight into the structure of KG.}
\notes{The classes form an ontology that can formally be represented as a poset.}
\notes{Consequently, triple types are also ordered in a poset.}
\notes{A KG includes triples that represent the values and triples that represent user-defined types that form the schema of a KG \cite{Savnik2025}.}
\notes{Denotational view of classes and type triples.}
\notes{The interpretations of classes and triple types form a poset based on the subset relation.}

% On stored or user-define typings of KG
\notes{Facts about the stored typings of the triples.}
\notes{In KGs we have types of individual objects represented as classes.}
\notes{Further, the types of the triples are the triples including types of triple components.}
\notes{A user-defined type of a triple is not linked directly to a triple.}
\nnotes{Types of triples are in a KG defined by specifying the types of identifiers that form triples.}
\nnotes{However, stored types of identifiers do not need to be those that appear in user-defined schema triple types.}
\nnotes{Hence, the types of ground triples must be derived from a KG by selecting the appropriate user-defined triple types from the types of components.}

%% Why types in KGs
\notes{What types are used for in a KG?}
\notes{Types can be used to verify the correctness of the ground triples and the structures that they form.}
\nnotes{The typing errors can appear in a KG if types of identifiers are specified incorrectly.}
\notes{Types define the context in a KG that allows placing (?) a structure of triples (sub-graph) in a KG.}
\notes{Disambiguation of property (predicate) placement. Later, binding of methods, etc.}
\nnotes{Typing the triple patterns is similar to typing ground triples \cite{Savnik2025}.}
\notes{Before we can define typing of languages that work with a KG, we have to be able to type a triple.}

%% About type checking in type theory
\notes{What is type checking? How it works...}
\notes{Types represent a higher-level description of ground triples.}
In type theory, types represent concepts that are used to classify the
values from a given language \cite{Pierce2002,Hindley1997}. Correct
typing assures that the functions are applied to correct parameters in
a program. The type-checking problem \\
\nnotes{Having a program verify that it conforms with the typing rules.}
\nnotes{In this process the rules can be applied in two directions.}
\nnotes{Type-assignment derives a type to an expression in a bottom-up manner.}
\nnotes{Here we use typing rules in the forward direction, hance type inference.}
\nnotes{Verifying that an expression adheres a given type uses typing rules in a backward direction.}
\nnotes{Here the rules decompose expressions into syntactic components and verifies recursively the types of comonents.}

% abstract of type checking KG ground triples
\notes{In order to check the types of a KG, the type of each ground triple has to be checked.}
\notes{There are two ways of computing the type of a triple.}
\notes{First, we can use types of identifiers representing the components of a triple.}
\nnotes{We call this type a \emph{ground type} since the types of identifiers classes from the bottom of the class ontology.}
\nnotes{Note that, because of the rule of subsumption \cite{Pierce2002}, the type of a ground triple is any triple type that is a supertype of a ground triple type.}
\notes{Second, we can select appropriate user-defined triple type $T$ such that the components of $T$ are the types of the components of the ground triple we are checking.}
\nnotes{The components of the user-defined triple types are noramlly from the top of the class ontology.}
\nnotes{Type checking of a ground triple is converted to checking the relationships between the ground type and user-defined type of a triple.} 
\notes{In all contexts we want to determine \emph{minimal type} of a triple, having the smallest interpretation.}}

% differences to classical TC and tying data (structures)
%\notes{On differences of type-checking KGs to classical type-checking problem.}
%\notes{On intersection and union types and their use in type-checking KGs \cite{Pierce2002,Dezani2020,Pierce1991}.}
%\notes{TC is search in the KG space.}
%\notes{TC is an algorithm for selecting the appropriate type of a triple.}
%\notes{TC does not need variables as in TC of computer languages.}
  
%% Relations to other approaches
%\notes{The problem is in between type checking and type inference.}
%\notes{Using stored types of idents to infer the type of an object (ground triple) and then check how it relates to stored types of triples. }
%\notes{The idea is close to bideriectional typing \cite{Dunfield2021} because of inferring and checking.}
%\notes{In KGs we first infer as much as possible and then check inferred type with the stored types.}
%\notes{...}

%% Abstract of type-checking method
%\notes{Type-checking of KGs (abstract).}
%\notes{Type-checking of ground triples from a KG.}
%\notes{Three phases of type-checking ground triples.}
%\notes{First, a lub type $T_{lub}$ of a ground triple is derived,}
%\notes{Second, a MIN type $T_{min}$ of the schema triple types is computed.}
%\notes{Finally, a sub-type relationship between the types $T_{lub}$ and $T_{min}$ is investigated.}
%\notes{The type $T_{lub}$ restricts $T_{min}$ in cases that the propery of $t$ has multiple different meanings.}
%\notes{Typing triple patterns and BGP queries in further work.}
%\notes{Identifying errors in typing of a KG.}
%----------------------------------------------------------------







\section{Preliminaries}


\subsection{Knowledge graph\label{sec:kg}}

This section defines a knowledge graph as a RDF graph \cite{rdf} using
RDF-Schema \cite{rdfschema} for the representation of the structural
part of a knowledge base. 

Let $I$ be a set of URI-s, $B$ be a set of blanks and $L$ be a set of
literals. Let us also define sets $S = I\cup\/B$, $P = I$, and
$O = I\cup\/B\cup\/L$. A \emph{RDF triple} is a triple
$(s,p,o)\in\/S\times\/P\times\/O$. A \emph{RDF graph}
$g\subseteq\/S\times\/P\times\/O$ is a set of triples. Set of all
graphs will be denoted as $G$.

The complete set of triples of a RDF graph is in the text denoted as
$\Delta$. We use $\Delta$ when we want to refer to the original set of
KG triples and not the data model of a KG presented in the following
Section.

%We say that RDF graph $g_1$ is \emph{sub-graph} of $g_2$, denoted
%$g_1 \sqsubseteq g_2$, if all triples in $g_1$ are also triples from
%$g_2$. 





\subsection{A data model of a KG}

To abstract away the details of the RDF data model we unify the
representation of knowledge graph by separating solely between the
identifiers and triples. The identifiers include the set of individual
identifiers and the set of class identifiers. The triples inlcude the
set of ground triples and the set of triple types.

%In view of the above formal representation of RDF triples, the
%complete set of identifiers is $\I=I\cup\/B\cup\/L$. The identifiers
%from $\I$ are classified into the sets including literals $\I_l$,
%individual (ground) identifiers $\I_i$, class identifiers $\I_c$,
%predicate identifiers $\I_p$.

However, since typing of a KG is based on the separation between the
values and the types, we use the following classification of
identifiers and triples. The set of values $\V$ includes the
individual identifiers $\V_i$ and the individual (ground) triples
$\V_t$. To be able to refer to the specific subsets of $\V_i$ in the
rules, we also introduce the set $\V_l\subseteq\V_i$, which denotes a
set of literal values, and the set $\V_p\subseteq\V_i$ that refers to
the set of predicates from a KG. Finally, a set $\T_p\in\T_i$ stands
for all predicates of a KG that now have the role of types. Note that
predicates are treated both as values and as types.

The set of all valid types $\T$ of a KG comprises the class
identifiers $\T_i$, i.e. types of individual identifiers $\V_i$, and
the triple types $\T_t$, i.e. the types of individual triples $\V_t$.

The set of class identifiers $\T_i$ related by subclass relation is a
poset forming a taxonomy of classes. Similarly also predicates, now
in the role of types, are ordered in a poset. The meaning of class
identifiers is established by their interpretations. The
interpretatation of a class identifier $c$ comprises the instances of
a given class $c$ es well as the instances of all $c$'s sub-classes. 

The individual triples include solely the individual identfifiers from
$\V_i$ in places of S and O, and predicates $\V_p$ in the place of P
component of a triple. The types of individual triples are product
types $S*p*O$ where $S,O\in\T_i$ and $p\in\T_p$. The product types are
in our data model of a KG written as a triple $(S,p,O)$. The triple
types are ordered by a subtype relation to form a poset. The subtype
relation among the triple types is defined on the basis of the subtype
relation among the classes and predicates. A more detailed formal
definition of a KG is given in \cite{Savnik2025}.





\subsection{Typing rule language}

\memo{
In this paper we define typing of a data language used to represent an
ABOX \cite{Brachman2004KnowledgeRR} of a knowledge base given in a
form of a knowledge graph. The data language specifys the assertions
in the form of ground triples (ABOX) and the schema of assertions as
the types of triples (TBOX). The ground triples are the instances of
the triple types that altogether define the schema of a KG.}

We do not use standard typing rule language
\cite{Pierce2002,Hindley1997} that includes a context $\Gamma$ where
the types of the variables are stored. We use a meta-language that is
rooted in first-order logic (abbr., FOL) but similar to the one used by
Pierce in \cite{Pierce2002} for the representation of records and
subtyping.

The rules are composed of a set of premises and a conclusion. The
premises can be expressions stating the set membership of an object,
typing and subtyping judgements, or expressions in the FOL including
the previous two forms. The expressions of FOL can express complex
premises such as the requirements for the LUB and GLB triple
types. The conclusion part of the rule is a typing or subtyping
judgment.

The symbols used in a rule are grounded by stating their membership in
the sets of identifiers ($\V$, $\V_i$ and $\V_t$) and triples ($\T$,
$\T_i$ and $\T_t$) defined in Section \ref{sec:kg}. When we write
$O\in\/S$ then we mean the \emph{existence} of $O$ in a set $S$. We
use universal quantification $\forall\/O\in\/S,\ p(O)$ when we state
that some property $p(O)$ holds for all objects $O$ from $S$. The
premises of the rule are treated from the left to the right. The
quantification of the symbols binds the symbols up to the last premise
unless defined differently by the parentheses.

Similar to \cite{Dunfield2021}, we differ between two interpretations
of rules. First, the \emph{generator} view of rules is the forward
interpretation where rules infer the types from the types derived by
premises. Second, the \emph{type-checking} view of the rules is the
backward interpretation. Given the expression and its type, the rules
are applied backwards to verify that the expression has that type.






\section{Typing identifiers}

%\memo{
%The set of identifiers $\I$ include ground identifiers $\I_g$, class
%identifiers $\I_c$, and the predicates (properties) $\I_p$ that are
%are both ground identifiers, since they are instances of rdf:Property,
%and similar to class identifiers, since they act as types and form an
%ontology of predicates.}

In this section we present the basic typing of ground identifiers
$\V_i$. The rules for typing ground identifiers are further used for
typing ground triples $\V_t$ in Section \ref{sec:triples}.  Typing of
literals $\V_l$ is described in Section \ref{sec:literals}. The rules
for deriving the stored types of ground identifiers are given in
Section \ref{sec:idents-stored}. Finally, the sub-typing relation
$\preceq$ is defined for the class identifiers $\T_i$ and the types of
ground identifiers are presented in Section \ref{sec:idents-typing}.

%However, before presenting the
%types of identifiers, we introduce the intersection and union types
%that are used for the description of the types of identifiers in the
%following Section \ref{sec:intsc-union}.

% \notes{Details.}
%\notes{1. First define base type of identifiers $:_\darr$ and stored subtyping relationship $\preceq_\darr$.}
%\notes{2. From the basis define the indent typing $:$ and subtyping rel $\preceq$ among identifiers.}
%\notes{3. Include the link between subtyping and typing.}
%\notes{4. Define lub type using $\land$ type for a given ground ident.}







\subsection{Typing literals\label{sec:literals}}

Literal values $\V_l\subseteq\V_i$ are the instances of literal types
$\T_l\subseteq\T_i$. The literal types $\V_l$ are provided by the
RDF-Schema dictionary \cite{rdfschema}. RDF-Schema defines a list of
literal types, such as xsd:integer, xsd:string, or xsd:boolean.

The literals are composed of literal values and literal types. For
example, the literal
"365"\textasciicircum\textasciicircum\/xsd:integer includes the
literal value "365" and a type xsd:integer. Typing of literals is
defined by the following rule.

\begin{equation}
\dfrac{L\in\V_l\quad T\in\T_l\quad \text{"L"\textasciicircum\textasciicircum\/T}\in\Delta}
      {L:T}  
\end{equation}

The rule states that a literal value $L$ is of a type $T$ if a literal
"L"\textasciicircum\textasciicircum\/T is an element of a KG
$\Delta$. A literal type $T$ is referencing a type from the RDF-Schema
dictionary.






\subsection{Ground typing and subtyping of identifiers\label{sec:idents-stored}}

The typing expression $V:_\darr\/T$ is a \emph{ground typing} relation
$:_\darr$ that links a value $V\in\V$ to a ground type $T\in\T$. The
ground typing relation is a one-step typing relation based on typing
stored in a KG.

In this section we deal with the identifiers $I\in\V_i$ which are the
values of type $T\in\T_i$. The ground types are directly linked to the
ground identifiers via a stored typing relation in the form
$(I,\text{rdf:type},T)\in\Delta$. The expression $I:_\darr\/T$ states that a
class identifier $T$ is a ground type of an individual identifier $I$.

The subtyping expression $T_1\preceq_\darr\/T_2$ defines a subtype
relationship between the types $T_1$ and $T_2$. In the case we deal
with the identifiers, the relation $T_1\preceq_\darr\/T_2$ denotes the
subclass relations since $T_1,T_2\in\T_i$. The relation $\preceq_\darr$ is
a one-step relation that is stored in $\Delta$. Note that a unique
notation for gound typing allows us to address differently the
\emph{stored} and the \emph{derived} types of a KG.

%\notes{Partial ordering defined with stored schema triples in a database.}
%\notes{The relationships that poset $\preceq_\darr$ covers are rdfs:subClassOf and rdfs:subPropertyOf. }
%\notes{Identifiers included in $:_\darr$ are between ground idents and base classes.}
%\notes{This allows us to separate and also address separately the ssg and subtyping relationship.}

\memo{Oportunity to introduce ``mixed'' objects including ground and schema components.}

The rule for the one-step typing relation $:_\darr$ is defined using the
predicate rdf:type.

\begin{equation}
\label{rul:ident-1step-type}
\dfrac{I\in\V_i\quad T\in\T_i\quad (I,\text{rdf:type},T)\in\D}
      {I :_\darr T}
\end{equation}

The individual entity $I$ can have more than one stored types. By
using Rule \ref{rul:ident-1step-type} as a generator, it synthesizes
all types $T_j^{j\in[1,n]}$ such that $I:_\darr\/T_j^{j\in[1,n]}$.

If we want to obtain all valid ground types of $I$, the rule can be
used either in some other rule that employs it as a generator, or we
can update above rule to generate a $\land$-type including all the
types of $I$ as presented in Section \ref{sec:intsc-union}.

The one-step subtyping relationship $\preceq_\darr$ is defined on
classes by using the RDF predicate rdfs:sub\-ClassOf as follows.

\begin{equation}
\label{rul:ident-1step-subtype}
\dfrac{C_1,C_2\in\T_i \quad (C_1,\text{rdfs:subClassOf},C_2)\in\D}
{C_1 \preceq_\darr\/C_2}
\end{equation}

The rule for the definition of the one-step subtyping relationship
$\preceq_\darr$ is based on the predicate rdfs:subPropertyOf.

\begin{equation}
\dfrac{P_1,P_2\in\T_p \quad (P_1,\text{rdfs:subPropertyOf},P_2)\in\D}
      {P_1\preceq_\darr\/P_2}
\end{equation}

The predicates have in the above rule the role of types in the sense
that they represent the names of the binary relations. 



\subsection{Typing and subtyping identifiers\label{sec:idents-typing}}

The one-step relationship $\preceq_\darr$ is extended with the
reflectivity, transitivity and antisymmetry to obtain the subtyping
relationship $\preceq$. The relation $\preceq$ forms a partial ordering of
class identifiers. The ground typing relation $:_\darr$ is then
extended with the \emph{rule of subsumption} presented as Rule
\ref{rul:typing-subsumption} to obtain a typing relation $:$.

First, the one-step relationship $\preceq_\darr$ is generalized to the
relationship $\preceq$ defined over class identifiers $\T_i$.

\begin{equation}
\label{rul:Ic-extended}
\dfrac{I_1,I_2\in\T_i \quad I_1\preceq_\darr\/I_2}
      {I_1 \preceq I_2}
\end{equation}

Next, the subtyping relationship $\preceq$ is reflexive.

\begin{equation}
\label{rul:Ti-reflexivity}
\dfrac{I\in\T_i}
      {I\preceq\/I}
\end{equation}

The subtype relationship is also transitive. 

\begin{equation}
\label{rul:Ti-transitivity}
\dfrac{I_1,I_2,I_3\in\T_i\quad\/I_1\preceq\/I_2\quad\/I_2\preceq\/I_3}
      {I_1\preceq\/I_3}   
\end{equation}

Finally, the subtype relationship is antisymmetric which is expressed
using the following rule.

\begin{equation}
\label{rul:Ti-antisymmetry}
\dfrac{I_1,I_2\in\T_i \quad I_1\preceq\/I_2 \quad I_2\preceq\/I_1}
      {I_1=I_2}   
\end{equation}

As a consequence of the rules
\ref{rul:Ti-reflexivity}-\ref{rul:Ti-antisymmetry} the relation
$\preceq$ is a poset.

Knowledge graphs include a special class $\top$ that represents the
root class of the ontology. In RDF ontologies $\top$ is usually
represented by the predicate owl:Thing \cite{Hoffart2013}. The
following rule specifys that all class identifiers are more specific
than $\top$.

\begin{equation}
\dfrac{\forall\/S\in\T_i}
      {S\preceq\top}
\end{equation}

%\subsection{Typing of identifiers.}
The one-step typing relation $:_\darr$ is now extended to the typing
relation $:$ that takes into account the subtyping relation $\preceq$.
The following rule states that a stored type is a type.

\begin{equation}
\dfrac{I\in\V_i\quad\/C\in\T_i\quad\/I:_\darr\/C}
      {I:C}
\end{equation}

The link between the typing relation and subtype relation is provided
by adding a typing rule called \emph{rule of subsumption}
\cite{Pierce2002}.

\begin{equation}
\label{rul:typing-subsumption}
\dfrac{I\in\T_i\quad\/S,T\in\T_i\quad\/I:S\quad\/S\preceq\/T}
      {I:T}    
\end{equation}


\memo{Properties have dual role: they are instances and types at the same time.}
\memo{Present the features of properties from this point of view.}






\section{Intersection and union types\label{sec:intsc-union}}

\memo{
The meaning of the $\land$ and $\lor$ types can be defined through
their interpretations. The following definition expresses the
denotation of a $\lor$ type with the interpretations of its component
types. Suppose we have a set of types
$\forall\/i\in\/[1..n],\ T_i\in\T$.

\begin{displaymath}
  \llbracket\lor[T_i^{i\in[1,n]}]\rrbracket_\D = \bigcup_{i\in[1..n]}\llbracket\/T_i\rrbracket_\D
\end{displaymath}

Similarly, the interpretation of a $\land$ type is the intersection of
the interpretations of its component types.

\begin{displaymath}
\llbracket\land[T_i^{i\in[1,n]}]\rrbracket_\D = \bigcap_{i\in[1..n]}\llbracket\/T_i\rrbracket_\D
\end{displaymath}}

\notes{CHECK: upper and lower bounds in the following rules.}

The meaning of the $\land$ and $\lor$ types can be seen through
their interpretations. The instances of the intersection type
$T_1\land\/T_2$ are objects belonging to both $T_1$ and $T_2$. The
type $T_1\land\/T_2$ is the greatest lower bound of the types $T_1$
and $T_2$. In general, $\land[T_i^{i\in[1,n]}]$ is the greatest lower
bound (abbr. GLB) of types $T_i^{i\in[1,n]}$
\cite{Pierce1991,Pierce1996}. The instances of the type
$\land[T_i^{i\in[1,n]}]$ form a maximal set of objects that belong to
all types $T_i$.

The rules for the $\land$ and $\lor$ types presented in this section
are general---they apply for the identifier types $\I_c$ and triple
types $\T_t$. The set of types $\T=\I_c\cup\T_t$ is used to ground
the types in the rules.

The interpretation of a type $\land[T_i^{i\in[1,n]}]$ in included in
interpretation of every particular type $T_i$. Hence, the type
$\land[T_i^{i\in[1,n]}]$ is the greatest lower bound of types
$T_i^{i\in[1,n]}$. This is stated by the following rule.

\begin{equation}
\label{rul:land-preceq-elems}
\dfrac{T_i^{i\in[1,n]}\in\T}
      {\land[T_i^{i\in[1,n]}] \preceq\/T_i^{i\in[1,n]}} 
\end{equation}

Further, the following rule states that if the type $S$ is a subtype
of every type $T_i^{i\in[1,n]}$ then $S$ is a subtype of
$\land[T_i^{i\in[1,n]}]$.

\begin{equation}
\label{rul:lbound-preceq-land}
\dfrac{S\in\T\quad\/T_i^{i\in[1,n]}\in\T\quad\/S\preceq\/T_i^{i\in[1,n]}}
      {S\preceq\land[T_i^{i\in[1,n]}]}  
\end{equation} 

The opposite to the above rule, the following rule states the
necessary conditions that must be met so that a type $T$ is a
supertype of a $\land$-type $\land[S_i^{i\in[1,n]}]$.
% this are upper bound types

\begin{equation}
\label{rul:land-preceq-ubound}
\dfrac{T\in\T\quad\/S_i^{i\in[1,n]}\in\T\quad\/S\in\{S_i^{i\in[1,n]}\}\quad\/S\preceq\/T}
      {\land[S_i^{i\in[1,n]}]\preceq\/T}  
\end{equation} 

The intersection and union types are dual. This can be seen also from
the duality of the rules for the $\land$ and $\lor$ types.

The instances from the union type $T_1\lor\/T_2$ are either the
instances of $T_1$ or $T_2$, or the instances of both
types. Therefore, $\lor[T_i^{i\in[1,n]}]$ is the least upper bound of
types $T_i^{i\in[1,n]}$ \cite{Pierce1991}.

\begin{equation}
\label{rul:lor-preceq-elems}
\dfrac{T_i^{i\in[1,n]}\in\T}
      {T_i^{i\in[1..n]}\preceq\/\lor[T_i^{i\in[1,n]}]}
\end{equation}

Finally, the following rules defines the necessary conditions to be
met for a type $T$ to be a supertype of a type
$\lor[S_i^{i\in[1,n]}]$.

\begin{equation}
\label{rul:lor-preceq-ubound}
\dfrac{T\in\T\quad\/S_i^{i\in[1..n]}\in\T\quad\/S_i^{i\in[1..n]}\preceq\/T}
      {\lor[S_i^{i\in[1..n]}]\preceq\/T}  
\end{equation}

Again, the opposite rule that defines the premises that must hold so
that $S$ is a subtype of a type $\lor[T_i^{i\in[1,n]}]$.

\begin{equation}
\label{rul:lbound-preceq-lor}
\dfrac{T\in\T\quad\/S_i^{i\in[1..n]}\in\T\quad\/S\in\{S_i^{i\in[1..n]}\}\quad\/T\preceq\/S}
      {T\preceq\lor[S_i^{i\in[1..n]}]}  
\end{equation}

Note that besides checking the subtype relation between a type treated
as a whole and some logical type, Rules
\ref{rul:land-preceq-elems}-\ref{rul:lbound-preceq-lor} can be used to
check the subtyping among arbitrary logical types.


\memo{
Note that the interpretation of a class $C$ is a set of instances
$\llbracket\/C\rrbracket_\D=\{I\ |\ I\in\I_i\land\/I:C\}$. Further, the
interpretation of a triple type $T$ is the set of ground triples
$\llbracket\/T\rrbracket_\D=\{t\ |\ t\in\T_t\land\/t:T\}$\footnote{Triple
  types $\T_t$ are presented in the following Section
  \ref{sec:triples}.}\cite{Savnik2025}.\\}

\memo{Put together the base types of ground identifiers using $\land$ type.}
\memo{First, the base type of an ground identifier is the $\land$ of all base types.}
\memo{The \emph{base type} of a ground identifier is defined explicitely!}





\subsection{The join and meet types\label{sec:join-meet-types}}

The $\lor$ and $\land$ types are logical types defined through the
sets of instances. Given two types $T$ and $S$ we have a least upper
bound $S\lor\/T$, and a greatest lower bound $S\land\/T$ types where
$S\lor\/T$ denotes a minimal set of objects that are of type $S$
\emph{or} $T$ (or both), and $S\land\/T$ denotes a maximal set of
objects that are of type $S$ \emph{and} $T$.

A KG includes a stored poset of classes and triple types that
represent types of the individual objects and ground triples. The
poset can be used to compute a join $S\sqcup\/T$ and a meet
$S\sqcap\/T$. Usual definition of the join and meet operators is by
using a least upper bound and a greatest lower bound if they exist
\cite{Pierce2002}, respectively. However, in a KG we are also
interested in the upper bound and lower bound types
\cite{DaveyPriestley2002}. Let us present an example.

\begin{example}
  Let $P=(U,\preceq)$ be a partialy oredered set $P$ such that
  $U=\{a,b,c,d,e\}$ and the relation
  $\preceq=\{a\preceq\/c,a\preceq\/d,b\preceq\/c,b\preceq\/d,c\preceq\/e,d\preceq\/e\}$.
  The upper bounds of $S=\{a,b\}$ are the elements $c$ and $d$. Since
  there is no lower upper bounds, the upper bounds $\{c,d\}$ are
  minimal upper bounds. The least upper bound of $S$ is $e$.

  In the case that we remove the element $e$ from $P$ then $P$ does
  not have a least upper bound but it still has two minimal upper
  bounds $c$ and $d$. $\finbox$
\end{example}

The least upper bound (abbr. LUB) is by definition one element. It has
to be related to all upper bounds via the relationship $\preceq$.  On
the other hand, the most interesting upper and lower bounds are
minimal upper bounds (abbr. MUB) and maximal lower bounds (abbr. MLB)
\cite{Knudstorp2024}. They are lower than the least upper bound and
higher than the greatest lower bound, respectively. They represent
more detailed information about the parameter set of types $S$ than
the LUB type of $S$.

The join $J=\sqcup[T_i^{i\in[1,n]}]$ is a set of MUB types
$J_j^{j\in[1,m]}\in\/J$ such that $J_j$ is an upper bound with
$T_i^{i\in[1,n]}\preceq\/J_j$, and there is no such $L$ where
$T_i^{i\in[1,n]}\preceq\/L$ without also having $J_j\preceq\/L$. Since
we have a top type $\top$ defined in a KG, the join of arbitrary two
types always exists.

The meet of types $T_i^{i\in[1,n]}]$, $M=\sqcap[T_i^{i\in[1,n]}]$, is
a set of the maximal lower bound types $M_j^{j\in[1,m]}\in\/M$ such
that $M_j$ is lower bound with $M_j\preceq\/T_i^{i\in[1,n]}$, and all
other lower bounds $U$ with $U\preceq\/T_i^{i\in[1,n]}$ entail
$U\preceq\/M_j$. Note that the meet of the set of types from a KG does
not always exist.

The join type is related to the $\lor$-type. Given a set of types
$\{T_i^{i\in[1,n]}\}$, the join $J=\sqcup[T_i^{i\in[1,n]}]$ is a set
of types $J_j^{j\in[1,m]}\in\/J$ that are the minimal upper bounds
such that $T_i\preceq\/J_j$ for $i\in[1,n]$. On the other hand,
Rule \ref{rul:lor-preceq-elems} for the $\lor$-types states
$T_i\preceq\lor[T_i^{i\in[1,n]}]$. However, the join type and
$\lor$-type differ in the interpretation.
$$\llb\lor[T_i^{i\in[1,n]}]\rrb_\D=\bigcup_{i\in[1..n]}\llb\/T_i\rrb_\D\subseteq\bigcup_{j\in[1,m]}\llb\/J_j\rrb_\D=\llb\sqcup[T_i^{i\in[1,n]}]\rrb_\D$$

While the interpretation of the type $\lor[T_i^{i\in[1,n]}]$ includes
precisely the instances of all $T_i$, the interpretation of the type
$\sqcup[T_i^{i\in[1,n]}]$ contains the instances of minimal upper bound
types. The interpretation of $\sqcup[T_i^{i\in[1,n]}]$ can include
interpretations of classes that are not among $T_i^{i\in[1,n]}$.

A meet type of $T_i^{i\in[1,n]}$ may not exist in a poset of types
from a KG. In general, the meet types $M=\sqcap[T_i^{i\in[1,n]}]$
exist in a class ontology if the types $T_i^{i\in[1,n]}$ are
\emph{bounded below} \cite{Pierce2002} which means that there exists a
type $L$ such that $L\preceq\/T_i$ for all $i$. The meet types are not
frequent on the lower levels of a class ontology from a KG.

As in the case of the $\lor$-type and the join type, the semantics of
the $\land$-type is similar to the semantics of the meet type. A
$\land$-type is a type that implements logical view of the greatest
lower bound type. Differently, the meet types are based on the
concrete poset of KG types and represent concrete types though their
interpretation is contained in the interpretation of a
$\land$-type. The type $\land[T_i^{i\in[1,n]}]$ denotes the
intersection $\bigcap\/\llb\/T_i\rrb_\D$ while the interpretation of a
meet type $M_j\in\sqcap[T_i^{i\in[1,n]}]$ includes the interpretation
of the meet types from $M$. Note that the instances of the meet types
are in the intersection of the instances of types
$T_i^{i\in[1,n]}$. The set $\bigcap\/\llb\/T_i\rrb_\D$ can also
include objects that are not instances of any meet type from
$M$. Hence,
$$\llb\land[T_i^{i\in[1,n]}]\rrb_\D=\bigcap_{i\in[1..n]}\/\llb\/T_i\rrb_\D\supseteq\bigcap_{j\in[1,m]}\llb\/M_j\rrb_\D=\llb\sqcap[T_i^{i\in[1,n]}]\rrb_\D.$$

In type-checking the ground triples, the join types are used in the
procedure for checking the types derived bottom-up against the stored
schema of a KG as presented in Section \ref{sec:3-ground-types}. The
join as well as meet types are useful in the procedure for
type-checking basic graph patterns \cite{Savnik2025a}. The $\lor$ and
$\land$-types are logical types that can be simplified in the typing
positions of a graph pattern by using typing rules, and can be
approximated by using join and meet types to obtain a more precise
concrete type of a graph pattern variable.




\subsection{Typing identifiers with $\land$ and $\lor$-types}

The $\land$ and $\lor$-types can model the available choices in
selection of the domain and range types of a triple type. The
available choices depend on the selected model (e.g., RDF-Schema). As
usual for the expressions including a variant of $\cup$ and $\cap$
operators more complex expressions can be transformed by moving
$\cup$ and $\cap$ either inside expression or towards the outside of
an expression. We define the rules for these transformations only if
they are needed for typing ground triples.

We start with gathering the ground types of a ground identifier
$I\in\V_i$. The ground types are gathered in the premise of the rule by using the typing
operation $:_\darr$ defined in Section \ref{sec:}

\notes{For $V\in\V$ gather ground types of identifiers with $\land$-type as $V:_{\darr}\land[T_i^{i\in[1,n]}]$.}
\nnotes{$V\in\V\quad\forall\/T_i\in\T,\ t:_{\darr}T_i$.}
\nnotes{The ground type of $V$ is a type $T_g=\land[T_i^{i\in[1,n]}]$.}
\ennotes{The following rule geathers all ground types of $V\in\V$.}

\begin{equation}
\label{rul:land-1}
\dfrac{I\in\V_i\quad\/T_i^{\s\/i=1..n}\in\T_i\quad\/I:_{\darr}T_i^{\s\/i=1..n}}
      {I:_{\darr}\land_{i=1}^n\/T_i}
\end{equation}

Let's have a look at $\land$ type composed of $V$'s ground types
$T_i^{i\in[1,n]}$ in the case $V\in\I_i$. In Yago \cite{Hoffart2013},
often $V$ has a set of very specific classes $C_s$ but also some
general classes $C_g$. The general classes $C_g$ are close to the
classes used in the schema triple types. If stored typing of $V$ is
correct, then $C_g$ includes the classes that are supertypes of
classes from $C_s$.

The ground type $\land[T_i^{i\in[1,n]}]$ can include pairs of types
$T_i\preceq\/T_k$ with $i\not=k$. Depending on the further use, we can
either compute the minimal or the maximal elements from the poset
$\{T_i^{i\in[1,n]}\}$ with respect to $\preceq$
\cite{DaveyPriestley2002}. The super-types of the minimal elements of
$\{T_i^{i\in[1,n]}\}$ include all valid types of $V$. Hence, we use
the set of minimal elements from $\{T_i^{i\in[1,n]}\}$ as the starting
point to explore the relations between the ground triple types and the
user-defined triple types of a triple $t$ including $V$ as a
component.

\memo{
the type $\land[T_i^{i\in[1,n]}]$ It makes sense
either to compute MIN or MAX of $\land[T_i^{i\in[1,n]}]$ yielding
$\land[S_1..S_m]$ where $m\le\/n$ and
$S_j^{j\in[1,m]}\in\{T_i^{i\in[1,n]}\}$. The operator MIN computes
$S_j^{j\in[1,m]}$ such that each $S_j$ is minimal among
$T_i^{i\in[1,n]}$. The operator MAX computes $S_j^{j\in[1,m]}$ such
that each $S_j$ is maximal among $T_i^{i\in[1,n]}$. Note that in both
cases there are no pairs among $S_j^{j\in[1,m]}$ related by $\preceq$.}

The operator MIN is defined on a poset of types
$(\{T_i^{i\in[1,n]}\},\preceq)$. Given a set of types $\{T_i^{i\in[1..n]}\}$
the MIN operator retains types $S_j^{j\in[1,m]}\in\{T_i^{i\in[1,n]}\}$
such that ${\nexists}\/T_k^{k\in[1,n]}(T_k\prec\/S_j)$. All pairs of
types $S_k,S_l\in\{S_j^{j\in[1,m]}\}$ with $k\not=l$ are
\emph{incomparable}, i.e.,
$S_1\not\sim\/S_2\equiv\/S_1\not\preceq\/S_2\land\/S_1\not\succeq\/S_2$.
The logical rule for the operation MIN is as follows.

\begin{equation}
\label{rul:min}
\dfrac{V\in\V\quad\/V:_\darr\land[T_i^{i\in[1,n]}]\quad\/S\in\/\{T_k^{k\in[1,n]}\}\quad\forall\/i\in[1,n],\ S\preceq\/T_i\lor\/S\not\sim\/T_i}
      {V:_{\Darr}S}
\end{equation}

The rule says that $S$ is a minimal type of a ground type
$\land[T_i^{i\in[1,n]}]$. $S$ is minimal since all other $T_i$ types are
either more general or equal ($\succeq$), or not related to $S$. The
rule generates all MIN types of $\land[T_i^{i\in[1,n]}]$.

The following rule is used for gathering all MIN types of
$\land[T_i^{i\in[1,n]}]$. The result is a conjunction of minimal types
$\land[S_1..S_m]$.

\begin{equation}
\label{rul:min-gather}
\dfrac{V\in\V\quad\forall\/i\in[1,m],\ V:_\Darr\/S_i}
      {V:_{\Darr}\land[S_j^{j\in[1,m]}]}
\end{equation}

Now we have a minimal ground type of a value $V$ in the form of a
conjunction of minimal types $S_i$ of $V$. The types that are
important from the perspective of typing computer languages defined on
values from a KG are the user-defined types of triples. The definition
of a user-defined triple type requires the knowledge about the meaning
of the binary relationship defined by a predicate---this is reflected
in the selection of the domain and range of the predicate.

The first step in verifying the relations between the ground and
user-defined types of a value $V$ is the computation of a join type
from a ground type of $V$. The join type of a ground type should be a
subtype of the user-defined types that describe the value $V$. If the
join type is not a subtype of the user-defined type then there is an
error in stored types of a value $V$.

Let us now present the typing rules that, given $V\in\V$,
determine the join of $\{T_i^{i\in[1,n]}\}$ as
$V:\sqcup[T_i^{i\in[1,n]}]$. Recall from Section
\ref{sec:join-meet-types} that we defined the operation join as the
minimal upper bounds of a set $\{T_i^{i\in[1,n]}\}$. Let
$P=(\T,\preceq)$ and $\{T_i^{i\in[1,n]}\}\subseteq\/P$. A join
$\sqcup[T_i^{i\in[1,n]}]$ is a set of minimal upper bounds
$\{S_j^{j\in[1,m]}\}$ that are related to all types $T_i^{i\in[1,n]}$
via $\preceq$, and are minimal.

%\nnotes{$\llb\land[T_i^{i\in[1,n]}]\rrb\subseteq\bigcup_{i\in[1..n]}\llb\/T_i\rrb\subseteq\llb\/T\rrb$.}
%\ennotes{A join type is used when we want to determine the user-defined type of $V$.}

Rules \ref{rul:join-logic}-\ref{rul:join-gather} present the logical
definition of the join type. The following Rule \ref{rul:join-logic}
determines one join type but can be used to generate all join types.

\begin{equation}
\label{rul:join-logic}
\dfrac{V\in\V,\, \/V:_\darr\land[T_i^{i\in[1,n]}]\quad\/S\in\T,\, T_i^{i\in[1,n]}\preceq\/S\quad\forall\/P\in\T,\, (T_i^{i\in[1,n]}\preceq\/P\land\/S\preceq\/P)\lor\/P\not\sim\/S}
      {V:_\sqcup\/S}
\end{equation}

The individual join types derived by the above rule are gathered
into one $\land$ type of join types by using the following rule.

\begin{equation}
\label{rul:join-gather}
\dfrac{V\in\V\quad\/S_i^{i\in[1,m]}\in\T\quad\forall\/i\in[1,m],\ V:_\sqcup\/S_i}
{V:_{\sqcup}\land[S_i^{i\in[1,m]}]}
\end{equation}

The join types are used as the starting point for searching the
correct user-defined type. In the case the predicate of has two
different definitions in two different contextes, then the paths from
join types to the domain and range classes determines the definition
of the triple type. The details are presented in Section
\ref{sec:stored-3types}.



%----------------------------------------------------------------
\memo{
The rule for filtering $\land[T_i^{i\in[1,n]}]$ of all
$T_i\succeq\/T_j$ where $i\not=j$ by using algorithmic typing is
defined as follows. The algorithm implementing the operation MIN is
presented in Section \ref{sec:alg-MIN}.

\begin{equation}
\label{rul:alg-min}
\dfrac{V\in\V\quad\vdasharr\/V:_\darr\land[T_i^{i\in[1,n]}]\quad\vdasharr\/\{S_j^{j\in[1,m}\}=\,\Darr\kern -3pt[T_i^{i\in[1,n]}]}
{\vdasharr\/V:_{\Darr}\land\/[S_j^{j\in[1,m]}]}
\end{equation}}

\memo{
The following Rule \ref{rul:join-alg} derives the complete join type
in one step. The operator join $\sqcup[T_i^{i\in[1,n]}]$ returns as a
result a set of minimal upper bound types of $T_i^{i\in[1,n]}$. Since
all the join types are valid types of $V$, we can group them into one
$\land$ type.

\begin{equation}
\label{rul:join-alg} 
\dfrac{V\in\V\quad\vdasharr\/V:_\darr\land[T_i^{i\in[1,n]}]\quad\/\vdasharr\{S_j^{j\in[1,m]}\}=\sqcup[T_i^{i\in[1,n]}]}
      {\vdasharr\/V:_\sqcup\land[S_j^{j\in[1,m]}]}
\end{equation}}

\memo{
%\notes{Integrate this explanation above.}
%\notes{Why computing a join type $S=\sqcup[S_1..S_n]$ of $\land[S_1..S_m]$?}
%\nnotes{Show how the $\land$ and $\sqcup$ types of $V$ are used to compute a type of $V$.}
%\nnotes{$V:_\sqcup\/S$ and there should be a path from $S$ to class components of schema triple types $T$.}
%\nnotes{... details about the above statement.}
%\nnotes{As such, $S$ is an approprite point to start searching stored types of $t$.}

\notes{Example from a KG.}
\notes{Let's have a look at $\sqcup$ types of $I$'s ground types $S_1..S_n$ from a KG. }
\nnotes{In many cases $I$ has a single type $S_1$ which is the same as the join type $S$.}
\nnotes{The super-classes of the join type $S$ have to be part of schema triple types}
\nnotes{to be selected as the types of subject or object.}
\nnotes{Often join type $S$ is close to the classes that are components of schema triple types.}
\notes{Can the situation with two super-predicates of a predicate $p$ lead to two different definitions of $p$?}}
%----------------------------------------------------------------












\section{Typing triples\label{sec:triples}}

{
\notes{We would like to type of a triple $t\in\T_i$.}
\notes{There are two basic aspects of a triple type.}
\nnotes{1. $t:T_g$ is computed bottom-up: from the stored types of triple components.}
\nnotes{2. $t:T_u$ can be computed from the user-defined types of properties.}
\nnotes{The relation $T_g\preceq\/T_u$ must hold if the typing of KG is correct.}
\nnotes{If the predicate $p$ of type $T_u$ is defined in multiple contexts, some of disjunctively}
\nnotes{linked components of $T_u$ may not be related to $T_g$.}
\nnotes{The filtering of $T_u$ is done by Rule \ref{rul:3-typing}.}

\notes{About the types that are computed bottom-up.}
\nnotes{Ground type of a triple is computed first by extending $:_\darr$ to triples.}
\nnotes{A triple can have multiple ground types $T_g=(\land[S_i^{i\in[1,k]}],p,\land[O_j^{j\in[1,m]}])$.}
\nnotes{Next, the join type $\sqcup[T_g]$ of ground type $T_g$ is derived.}
\nnotes{The type $\sqcup[T_g]$ is used as a stepping stone to determine the final type of $t$.}
\nnnotes{To be used for type-checking graph patterns.}

\notes{Schema triple types are user-defined types.}
\nnotes{Stored types for a predicate $p$ are defined via the predicates rdfs:domain and rdfs:range.}
\nnotes{From the top of the ontology, the stored type $:_\uarr$ is determined based on $p$.}
\nnotes{However, given $p$ we can have a triple type $(T,p,S)$ such that $T$ or $S$ are}
\anotes{defined for some $p'\succeq\/p$.}
\nnotes{Special case: $p'$ has two domains $T^1$ and $T^2$---type is then}
\anotes{$(T^1\lor\/T^2,p,\_)\equiv\/(T^1,p,\_)\lor(T^2,p,\_)$.}
\nnotes{$\T_\uarr=\{(T,p,S)| p\in\I_p\land(p,\text{rdfs:domain},T)\in\Delta\land(p,\text{rdfs:range},S)\in\Delta\}$} 
\nnotes{Derived types of the stored types are computed using Rule \ref{rul:typing-subsumption}.}
\nnotes{Derived types of $\T_\uarr$ include the complete top of the ontology }

\notes{Schema triple types for a given predicate $p$ are computed as MIN of valid stored types for $p$.}
\nnotes{The MIN types of types obtained using $:_\uarr$ are the smallest triple types}
\nnotes{including MIN classes as components.}
\nnotes{Stored type have to be minimal to have minimal interpretaton (e.g., type of a triple pattern).}
\nnotes{Finally, the type $:$ of $t$ is deterimned by summing alternative $:_\Darr$ types.}
\nnotes{Note there can be more than one $\Darr$-type.}
\nnotes{This happens when triple types include property that is defined in two different contexts.}

\notes{Interactions between the $\land/\lor$ types of triple components and triples must be added.}
\nnotes{Analogy between the types of functions in LC and types of triples.}
\nnotes{Show rules relating $\land/\lor$ types and triple types. Example.}
\nnotes{E.g., $(S_1\land\/S_2)*p*R = S_1*p*R\land\/S_2*p*R$.}
\nnotes{Are all rules covered?}

\notes{Predicates should be treated in the same way as the classes.}
\nnotes{They can have a rich hierarchy.}
\nnotes{\emph{Note:} Discussion on special role of predicates and their relations to classes?}
\nnotes{Mention Cyc as the practical KB with rich hierarchy of predicates.}}
%----------------------------------------------------------------






\subsection{Triple types\label{sec:3-types}}

A type of a triple $(s,p,o)\in\V_t$ is a triple $(D,p,R)\in\T_t$ such that
$s:D$ and $o:R$ holds. The types $D$ and $R$ are type expressions that
represent either a class or a $\land$-type composed of classes.

The $\land$-types reflect the semantics of RDF-Schema \cite{rdfschema}
which permits the definition of multiple domains and ranges of the
predicate $p$. Consequently, each predicate $p$ has exactly one triple
type of the form
$$(\land_{i=1}^n\/D_i^{\s\/i=1..n},p,\land_{j=1}^m\/R_j).$$

\begin{example}
  If $p$ has two domains $D_1$ and $D_2$, and a single
range type $R$ then the type corresponding to $p$ is
$(D_1\land\/D_2,p,R)$. $\finbox$
\end{example}

\noindent
The interpretation of a triple type $(D,p,R)$ is defined as
follows.
$$\llb(D,p,R)\rrb_\Delta=\{(s,p,o)\mid\/s\in\llb\/D\rrb_\Delta\land\/o\in\llb\/R\rrb_\Delta \}$$

The subtype relationship among the triples is defined on the basis of
the subtype relationship among the classes and the $\land$ and
$\lor$-types defined on classes, and predicates. The following rule
defines the relationship $\preceq$ beween two triple types.

\begin{equation}
\label{rul:3-subtype}
\dfrac{T_1\in\T_t,\ T_1=(D_1,p_1,R_1)\quad T_2\in\T_t,\ T_2=(D_2,p_2,R_2)\quad\/D_1\preceq\/D_2\quad\/p_1\preceq\/p_2\quad\/R_1\preceq\/R_2}
      {T_1\preceq\/T_2}
\end{equation}

It is obvious from the definition of the subtyping relationship among
the triple types that
$\llb\/T_1\rrb_\Delta\subseteq\llb\/T_2\rrb_\Delta$. Note that Rule
\ref{rul:3-subtype} handles the $\land$-types of the subject and
object through Rules
\ref{rul:land-preceq-elems}-\ref{rul:land-preceq-ubound}.







\subsection{Ground types of a triple\label{sec:3-ground-types}}
% checked

\noindent
The ground types of a triple $t$ are either a stored ground type, a
minimal ground type, or a join ground type. The stored ground type
includes types that are stored in a KG. The minimal ground type then
consists solely of the minimal types from the stored ground
types. Finally, the join ground type is the conjunction of minimal
upper bound types \cite{Knudstorp2024}.
 
A ground type of an individual identifier $I$ is a class $C$ related
to $I$ by one-step type relationship $:_\darr$, as presented by Rule
\ref{rul:ident-1step-type}. In terms of the concepts of a knowledge
graph, $C$ and $I$ are related by the relationship rdf:type.

A ground type of a triple $t=(I_s,p,I_o)$ is a product type
$T*p*S$ that we represent as a triple $(T,p,S)$. A triple type
includes the ground types of $t$'s components $I_s$ and $I_o$, and the
property $p$, which now has the role of a type. A ground type of a
triple is defined by the following rule.

\begin{equation}
\label{rul:3-ground}
\dfrac{t\in\V_t,\ t=(I_s,p,I_o)\quad T,S\in\T_i\quad I_s:_\darr\/T\quad I_o:_\darr\/S\quad p:_\darr\text{rdf:Property}}
      {t:_\darr(T,p,S)}
\end{equation}

The type $T$ is either a class identifier or a $\land$-type composed
of a conjunction of class identifiers. The predicates are treated
differently from the subject and object components of triples. The
predicates have the role of types while they are instances of
rdf:Property.

The minimal ground type of a triple $t$ can be obtained by using the
minimal ground types of the triple components. 

\begin{equation}
\label{rul:3-min-ground}
\dfrac{t\in\V_t,\, t=(I_s,p,I_o)\quad T,S\in\T_i\quad I_s:_\Darr\/T\quad I_o:_\Darr\/S\quad p:_\darr\text{rdf:Property}}
      {t:_\darr\/(T,p,S)}
\end{equation}

The component types $T$ and $S$ of the minimal ground type
$(T,p,S)$ can represent $\land$-types. The following rule transforms
a triple type including $\land$-types into a $\land$-type of simple
triple types composed of class identifiers in place of $S$ and $O$
components. The rule is expressed in a general form by using the
typing relation "$:$", which can be replaced by any labeled typing
relation.

\begin{equation}
\label{rul:3-land-transform}
\dfrac{t\in\V_t\quad t:(\land_{i=1}^n\/S_i,p,\land_{j=1}^mR_j)}
      {t:\bigwedge_{i=1..n, j=1..m}(S_i,p,R_j)}
\end{equation}

The type in the conclusion of the rule is constructed by the Cartesian
product of the sets of types belonging to types of $S$ and $O$
components. Since each of the types $S_i$ and $R_j$ is valid for the
components $S$ and $O$, respectively, then also the triple types from
the conclusion of the rule are valid.

The above decomposition of a triple type into a set of triple types is
useful when we check the ground types against the schema triple types
to select the valid schema triple type of a triple. This is detailed in
Section \ref{sec:stored-3types}.

Finally, the join of a set of ground types is a set of minimal upper
bound types. Similarly to the previous two rules, the join is defined
on the basis of joins of triple type components $S$ and $O$.

\begin{equation}
\label{rul:3-join}
\dfrac{t\in\V_t,\ t=(I_s,p,I_o)\quad T,S\in\T_i\quad I_s:_\sqcup\/T\quad I_o:_\sqcup\/S\quad p:_\darr\text{rdf:Property}}
      {t:_\darr\/(T,p,S)}
\end{equation}

The join type $(T,p,S)$ includes in the components $S$ and $O$ the
$\land$-types comprising one or multiple MUB classes. When we convert
this type into a conjunction of single MUB triple types, then each MUB
triple type stands for all ground triple types of~$t$.

We can easily see that each particular triple type is an MUB type
since it includes MUB types in its components. Since the MUB types of
the components are incomparable by $\preceq$ then also the MUB triple
types obtained by Rule \ref{rul:3-join} are incomparable.

All rules defined for the ground triple types rely on inferring the
types of their components. The reason for this are the stored typing
of individual identifiers as well as the stored poset relation
$\preceq$, which are solely defined on classes.






\subsection{Schema triple types\label{sec:stored-3types}}
% checked L

The schema types are types of triples defined by a variant of KG
schema. The schema definition language currently used in KGs is either
RDF-Schema \cite{rdfschema} or RDF-Schema combined with OWL
\cite{owl2} vocabulary. In this section, we present the semantics of
both approaches.

We do not expect that the domain and range of the predicate are defined
for each particular predicate $p$. They can be inherited from the
super-predicates of $p$. Hence, a predicate $p$ has the domain and
range defined either directly, when domain and range are defined for
the predicate $p$, or indirectly, if the domain and range are defined
for $p$'s super-predicates and inherited by the predicate $p$.

The rules for the derivation of the schema triple type of a given
triple $t$ are presented for two different schema definition
languages. In Section \ref{sec:stored-3types}.1 we present the
derivation of schema types when the RDF-Schema is used. Further, in
Section \ref{sec:stored-3types}.2 we define the typing rules for
KGs that use RDF-Schema together with $\land$ and $\lor$ types.






\subsubsection{\ref{sec:stored-3types}.1 KGs with RDF-Schema.}
When a KG is restricted by using RDF-Schema, we can specify one or more
domain and range types. The semantics of RDF-Schema
\cite{rdfsemantics} interprets multiple domains and ranges with
$\land$-type. If $p$ has two domains $T_1$ and $T_2$ then $p$ can link
subjects $I$ that are of type $T_1$ \emph{and} $T_2$. The domain type
of $p$ is then $T_1\land\/T_2$, or, in terms of OWL \cite{owl},
owl:intersectionOf$(T_1\ T_2)$. The RDF-Schema does not allow the
definition of the domain of a predicate with the
$\lor$-type. Consequently, each predicate can have only one meaning.

We first determine \emph{all} valid schema types for a given triple
$t=(s,p,o)$. A schema triple type comprises a predicate $p$ and the domain and range types
linked to predicates $p'\succeq\/p$. The domain and range types can be
defined for the predicate $p$ and/or inherited from the predicates
$p'\succ\/p$. In addition, the domain and range types can be, in
general, inherited from two different super-predicates of $p$.

\begin{equation}
\label{rul:3-stored1}
\dfrac{\begin{array}{c}
       t\in\V_i,\, t=(s,p,o)\quad p_1,p_2\in\T_p,\, p\preceq\/p_1\preceq\/p_2\quad \\
       T_i^{\s\/i=1..n}\in\T,\, (p_1,\text{domain},T_i)\in\/\Delta\quad S_j^{\s\/j=1..m}\in\T,\, (p_2,\text{range},S_j)\in\Delta
       \end{array}}
      {t:_\uarr\/(\bigwedge_{i=1}^n\/T_i, p, \bigwedge_{j=1}^m\/S_j)}
\end{equation}

The above rule generates pairs of types of the domain and range of a
predicate $p$. We allow that a domain and range types are defined for
different predicates $p_1,p_2\preceq\/p$ since such a situation can
appear in a KG. However, we have to be careful that the rules of
inheritance are respected. We can only inherit from $p_1$ and $p_2$
that are related by subtype relation:
$p\preceq\/p_1\preceq\/p_2$. This condition restricts the domain and
range to be defined on the same path from $p$ to some maximal element
$m$ of the predicate $p$ poset. 

The above Rule \ref{rul:3-stored1} generates all valid schema types of
a triple $t$. From the set of all valid stored types of $t$ we select
the subset including only the minimal types. The following rule is a
logical judgment for a minimal schema type of a $t$. 

\begin{equation}
\label{rul:min-stored-type}
\dfrac{t\in\T_i\quad T\in\T,\, t:_\uarr\/T\quad\/S_i^{\s\/i\in[1,n]}\in\T,\, t:_\uarr\/S_i\quad T\preceq\/S_i\lor\/T\not\sim\/S_i}
      {t:_{\uarr}\/T}
\end{equation}

The first part of the premise says that $t$ is a ground triple and there
exists $T\in\T$ which is a type of $t$. The second part of the premise
requires that $T$ is the minimal type of all types $S$ of $t$. In
other words, there is no type $S_i^{i\in[1,n]}$ of $t$ that is a
subtype of $T$. Hence, $T$ is the minimal type of the stored triple
types of $t$.

Note that if the schema is defined by using RDF-Schema, and the stored
schema typing is correct, then the condition $T\sim\/S_i$ is always
$true$, and Rule \ref{rul:min-stored-type} generates exactly one
minimal type. Furthermore, under the restrictions of RDF-Schema, we can
not define a predicate $p$ with two meanings. If we were to specify two
different domains or ranges of $p$, then the reasoner would treat the
domain and range types as $\land$-types. Each instance of the domain
(range) type has to be an instance of all specified types of the
domain (range).

% 



\subsubsection{\ref{sec:stored-3types}.2 KGs with the contextual representation.}

The collective findings of the research in the area of Cognitive
Science \cite{Hollister2017} shows that natural language is inherently
contextual, and the context is essential in the human representation
of knowledge and reasoning. 

While the current trend is to enforce exactly one meaning of a
predicate in KGs, the contextual representation and reasoning allow
the definition of multiple senses of a predicate. There are many
motivations for adopting contextual representation and reasoning in a
KG. First, with the evolution of KGs, there are many examples where a
KG is represented in a modular way, splitting the dataset into parts
that correspond to the contexts. The meaning of a predicate can be
different in different contexts, while the reasoner is able to
disambiguate among the different meanings of a predicate.
The practical examples of KGs using contexts include the named graphs
in DBpedia \cite{Auer2007}, Wikidata \cite{vrandecic2014}, and Yago
\cite{Hoffart2013}. Another example is Cyc \cite{cyc} that uses
microtheories to represent different contexts. Similar to Cyc, Scone
\cite{Fahlman2011} is a KR system that can define spaces (contexts)
and uses contextual reasoning.

The second motivation for using contextual representation of KGs is
the problem of a predicate with multiple senses represented with
multiple sub-predicates. In a query---whether expressed in natural
language, logic or as a database query---it is difficult to
disambiguate the correct sub-predicate for the particular query. The
alternative is that a user must explicitly select a correct sense of
a predicate (i.e., a sub-predicate) in the query.

Finally, a predicate can be compared to a mathematical function since
it represents a binary relation. In mathematics, a function is not
represented by its name only, but with a function type including,
besides the function name, also the types of its domain and
range. Types of functions can disambiguate among the different
functions with the same name but different domain and range
types. Similarly, the types can be effectively used to disambiguate
the meaning of the predicate in a KG.

To be able to study the behavior of KG predicates with multiple senses
in the presence of triple types, we propose a minimal KR schema
language that includes the triple types stored in a KG as triples. In
case there is more than one triple type including a predicate $p$,
then these are treated as alternatives. For example, if a KG includes
triples $(T_1,p,T_2)$ and $(T_3,p,T_4)$, where
$T_j^{\s\/j=1..4}\in\T_i$, then the type of ground triples including
$p$ is $T_1*p*T_2\lor\/T_3*p*T_4$.

Further, the proposed KR language can use $\land$ and $\lor$-types in
subject and object components of triple types. The $\land$ and
$\lor$-types are often implemented in KGs in the form of OWL type
constructors owl:intersectionOf and owl:unionOf\footnote{The OWL union
  and intersection type constructors are employed mostly in domain-specific KGs like biomedical and genomic ontologies. In these
  scientific fields, the knowledge base includes large ontologies where
  new classes can be defined as logical combinations of existing
  classes.} The use of $\lor$-type in place of the domain or range
type is redundant since it can be expressed with multiple triple
types. Hence, the \emph{minimal KR schema language} includes solely
the triple types of the form
$$\textstyle\bigwedge_{i=1}^n\/D_i*p*\bigwedge_{j=1}^m\/R_j.$$

Let us now present the rules that derive the stored types of a ground
triple $t$ in the case our minimal KR schema language is used for the
definition of the KG schema. First, the following Rule
\ref{rul:3-stored2} generates all valid alternatives of stored triple
types given a triple $t$. Note that $D_i$ and $R_i$ are the types of
domains and ranges of $p$ that can stand for a $\land$-type.

\begin{equation}
\label{rul:3-stored2}
\dfrac{t\in\V_t,\, t=(s,p,o)\quad T_i^{\s\/i=1..n}\in\T_t,\, T_i=(D_i,p,R_i)\quad\/T_i^{\s\/i=1..n}\in\Delta}
      {t:_\uarr\/\bigvee_{i=1}^n\/T_i}
\end{equation}

A schema triple type $\bigvee_{i=1}^nT_i$ of $t$ is a disjunction
of all valid schema triple types of $t$. Similar to Rules
\ref{rul:min}-\ref{rul:min-gather}, which are defined to find minimal
types of a set of classes, the following Rules
\ref{rul:3-min}-\ref{rul:3-min-gather} generate the set of minimal
schema triple types of a set, including all valid schema types of $t$.

\begin{equation}
\label{rul:3-min}
\dfrac{t\in\V_t\quad\/t:_\uarr\bigvee_{i=1}^n\/T_i\quad\/S\in\/\{T_i\}_{i=1}^n\quad\forall\/i=1..n,\, S\preceq\/T_i\lor\/S\not\sim\/T_i}
      {t:_{\Darr}S}
\end{equation}

The rule says that $S$ is a minimal schema type of a schema type
$\bigvee_{i=1}^n\/T_i$. $S$ is minimal since all other types $T_i$ are
either more general or equal ($\succeq$), or not related to $S$. The
following rule gathers all minimal schema types of
$\bigvee_{i=1}^n\/T_i$.

\begin{equation}
\label{rul:3-min-gather}
\dfrac{t\in\V_t\quad\forall\/j=1..m,\, t:_\Darr\/S_j}
      {t:_{\uarr}\bigvee_{j=1}^m\/S_j}
\end{equation}

The above rule is identical to the Rule \ref{rul:min-gather} except
that in this setting it handles triples and not identifiers. The
result is a disjunction of minimal schema types $S_1,\ldots,S_m$.






\subsection{Typing a triple}

Before we present the final step in typing a triple $t$, an overview
of the work done so far is given. First, we derive the ground type of
the triple $t$ and then infer the minimal upper bound of the ground
type. The ground typing inspects all ground types of $t$'s
components. In case the MUB type is close to $\top$ type then there is
an outlier in the set of ground types, it can be revealed in
computation of the MUB type. 

Second, the schema type of $t$ is derived from the schema of a KG. In
case we use RDF-Schema semantics of a KG then the rules infer a single
minimal schema type. If there are more than one schema types than the
domains and ranges from the different definitions are merged into one
$\land$-type. At this point we can not verify if there are any errors
in types from the domain and/or range of a predicate. in schema
(stored) typing. In case we use RDF-Schema with $\land$ and $\lor$
types, then the rules can infer multiple disjunctive schema
types. Also in this case there are no additional constraints that
could be verified.

Let us now present final typing of $t$ by relating the ground type and
schema types (minimal schema triple types) via subtyping relation. If
the RDF-Schema semantics is used then we have a single schema type
which should be related to the ground type of $t$. The following Rule
\ref{rul:3-typing-one} derives the final type of $t$ under RDF-Schema
semantics. 

\begin{equation}
\label{rul:3-typing-one}
\dfrac{t\in\T_i\quad T\in\T_t,\, t:_{\darr}T\quad S\in\T_t,\, t:_\uarr\/S\quad T\preceq\/S}
      {t:S}
\end{equation}

The type of $t$ is computed by first deriving the ground type $T$ and
the schema type $S$ of $t$. $S$ is the final type of $t$ if $T$ is a
subtype of $S$. In case $T\not\preceq\/S$ then the ground type
$T=\land[T_i^{i\in[1,n}]$ includes at least one $T_i\not\preceq\/S$.

\begin{equation}
\label{rul:3-typing-one}
\dfrac{t\in\T_i\quad t:_{\darr}T\quad\/t:_\uarr\bigvee_{j=1}^{m}\/S_j}
      {t:\bigvee\{S_j\mid\/j\in[1,m], T\preceq\/S_j\}}
\end{equation}



... \\




%----------------------------------------------------------------
\memo{
\notes{Why using $\land$ and $\sqcap$ types for typing a triple $t$?}
\nnotes{We would like to check typing of a triple $t\in\T_i$.}
\nnotes{We compute first the ground type $T_g=\land[T_i^{i\in[1,n]}]$ and a stored type $T$ of $t$.}
\nnotes{The ground type $T_g$ is computed from the ground types of $t$'s components.}
\nnotes{The subtype relation should hold $T_g\preceq\/T$.}

\notes{Two ways of defining semantics.}
\notes{1) enumeration style: stored types are enumerated as alternatives ($\lor$).}
\notes{2) packed together: alternative types are packed in one $\bigvee$ type.}
\notes{One advantage of (1) is that individual glb types can be processed further individually.}
\notes{Advantage of (2) is the higer-level semantics without going in implementation.}

\notes{Stored types have to be related to all join ground types to represent the correct type of a triple.}
\notes{It seems it would be easier to check the pairs one-by-one using (1) in algorithms.}
\notes{In case of using complete types in the phases, types would further have to be processed by $\land,\lor$ rules.}}

\memo{
\medskip
\notes{How to compute $T\preceq\/S$? Refer to position where we have a description.}
\notes{Order the possible derivations, gatherings (groupings) ... of types.}

\notes{Possible diagnoses.}
\notes{Components not related to a top type of a triple?}
\notes{Components related to sub-types of a top type?}
\notes{Above pertain to all components.}}

\memo{
\subsection{Typing a graph}

\notes{What is a type of a graph?}
\nnotes{A type of a graph is a graph!}
\nnotes{It includes a set of triple types forming a schema graph.}

\notes{Typing schema triples?}
\nnotes{What can be checked?}
\nnotes{Is a schema triple properly related to the super-classes and types of components.}
\nnotes{Consistency of the placement of a class in an ontology.}
\nnotes{A class or predicate component not related to other classes?}
\nnotes{A class or predicate component attached to ``conflicting'' set of classes? ?}
\nnotes{Any other examples?}

\notes{Typing a graph.}
\nnotes{Checking whether triple types match in the meeting points.}
\nnotes{What is the type in meeting points of two triple types?}
\nnotes{Since a tye of a graph should present any legal triple in $\D$}
}
%----------------------------------------------------------------




\section{Implementation}





\section{Related work}

Most of the related work is cited in the text when presenting the
particular topic. In this section, we present only the work that is not
directly related to the presented work but represents a contribution
to the typing of the knowledge graphs.

The entity typing and type inference deal with predicting and
inferring the types (classes) of entities that are either missing or
incorrect. The automatic type assignment can use logic-based
assignment where a reasoner infers the type of an entity from the
schema \cite{Horrocks2003}. Alternatively, the rule-based inference
can be employed on the rules defined as a schema by knowledge
engineers \cite{Horrocks2004}. Reasoners use them
automatically. Another alternative is the use of ML-based type
prediction \cite{Yaghoobzadeh2018}. Various techniques can be used, like entity embeddings and graph neural networks, to produce embeddings
to be fed into the classifier.

The schema-based type checking is about the verification of RDF-Schema
\cite{rdfschema12} and OWL \cite{owl2} rules and constraints against
the data (ABox) and schema (TBox) parts of a KG
\cite{Baader2002,Horrocks2003,OWL2Spec2012}. The domain and range
types of predicates are checked to determine whether they are correctly interpreted
among the ground triples of a KG. The consistency of subtyping and
inheritance is verified in the KG schema and in the data. Similarly,
the disjointness of types and other OWL constraints is
checked. Most of the presented themes are covered by tools based on
SHACL \cite{shacl2017} and ShEx \cite{shex2015}.

Type checking in query answering ensures that the variables and
results of a query over a KG are consistent with the schema of a KG
\cite{Zhao2017,Zhang2019}. A type-checker for queries requires that a
query respects class hierarchies, domain/range constraints, and
disjointness rules. The type information can also be used to improve
the query optimization and execution. In \cite{Kollia2013}, they
propose to leverage type information to optimize query execution and
filter semantically invalid results. In most cases, the presented
approaches use fragments of types that are adapted for the particular
problem.

% Intersection / Union types in schema
% Typed embeddings / type-aware representation learning
% Typing for relation / predicate sense disambiguation

\memo{
\notes{Comparing typing relation in an OO model with a KG \cite{Pierce2002}.}
\notes{The values from KGs have similar structure to the values of object-oriented models.}
\notes{However, the predicates of a KG are more expressive than the data members of the classes.}
\notes{Similarly, the record types form a lattice under subtype relationship with least upper bound and greatest lower bound based on sets of record attributes.}

\notes{Include the differences between Pierce's (classical) sub-typing view of stored sub-class relationships among classes and the approach taken in this paper}
\notes{Pierce treats classes as generators of objects that inherit methods and data members from its super-class.}
\notes{The methods are inherited by copying the definitions in each subclass and then explicitely calling the method in the superclass.}
\notes{List the differences: classes are identifiers, there is a sub-class relationship included in a sub-typing relationship.}}





\section{Conclusions}

Typing of a knowledge graph can serve many theoretical and practical
purposes. First, typing of a KG can be extended to type-checking that
can identify inconsistencies in a KG. Next, typing a KG can be used as
the basis for typing basic graph patterns and SparQL, and with this,
can be used in tasks such as query optimization and
reasoning. Finally, we show in the paper how types can be employed to
disambiguate the sense of a predicate.

... \\





% \section{Acknowledgments}
%
%The authors acknowledge the financial support from the Slovenian Research Agency (research core funding No. P1-00383).

\bibliographystyle{abbrv}
\bibliography{biblio}

\end{document}



